\documentclass{article}
\usepackage[utf8]{inputenc}
\usepackage{graphicx}
\title{An example of using \LaTeX{} and Gnuplot: the Gamma function}
\author{eliaszap93 }
\date{\today}

\begin{document}
\maketitle
\noindent In mathematics, the gamma function is one commonly used extension of the factorial function to complex numbers. 
 The gamma function is defined for all complex numbers except the non-positive integers. 
 For any positive integer $n$,
\begin{equation}
    \Gamma (n)=(n-1)!\ .
\end{equation}
Derived by Daniel Bernoulli, for complex numbers with a positive real part the gamma function is defined via a convergent improper integral:
\begin{equation}
    \Gamma (z)=\int _{0}^{\infty }x^{z-1}e^{-x} dx, \qquad \Re (z)>0
\end{equation}

\begin{figure}
    \centering
    \input{plot-gamma.tex}
    \caption{this is a caption}
\end{figure}

\end{document}